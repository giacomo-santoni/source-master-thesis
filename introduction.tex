Neutrinos are the most abundant known matter particles in the Universe. Although they appear in the Standard Model as massless particles, the evidence of the flavour oscillations implies a non-zero neutrino mass. Experimental studies of this phenomenon have led to determine many of the parameters linked to neutrino masses, by using several complementary channels and sources. However, the determination of some parameters, such as the ordering of neutrino masses, the CP violating phase and the value of $\theta_{23}$ mixing angle, represents a still open question. Next generation experiments are being built to search for these unknowns and to find which is the most suitable model for their description.

The Deep Underground Neutrino Experiment (DUNE) will be an accelerator-based experiment that will utilize the highest power neutrino beam, peaked at 2.5 GeV, a Near Detector at Fermilab, and a Far Detector at the Sanford underground laboratories in South Dakota, $\sim$ 1300 km away. It will measure the neutrino mass ordering, the CP violation phase, and the consistency of the three-flavour paradigm.
The Far Detector will consist of four Liquid Argon Time Projection Chambers, with an overall mass of 68 ktons, which allow the identification of several neutrino processes. At the near site a detector complex will contribute to the analysis of the Far Detector data, by providing complementary information with measurements on the neutrino beam on- and off-axis and by refining the neutrino interaction models.

At the Near Detector complex, SAND is a multipurpose detector composed of a superconducting solenoidal magnet that surrounds a calorimeter, repurposed from the KLOE experiment at the INFN Frascati laboratories. An inner tracker and an active $\sim 1$ ton liquid argon target (GRAIN) are placed inside the magnetic volume. SAND will continuously monitor the beam by performing tracking and calorimetric measurements of neutrino interactions. GRAIN and the downstream tracker will also contribute to neutrino interaction model studies and to constrain nuclear effects, by providing a large sample of neutrino interactions on different nuclear targets.

The GRAIN sub-detector is equipped with \emph{coded aperture cameras} and relies on a novel technique: the imaging of the argon scintillation light produced by the passage of a charged particle. A camera is composed of a coded aperture mask, which has a specific pattern of holes, and a SiPM matrix as an image sensor. Thanks to the geometric properties of the coded mask, the voxelized region of photon emission can be estimated from the detected image with a suitable algorithm. 

Starting from this voxel distribution, this work aims to assess the performance of a track finding algorithm for the reconstruction of charged-current quasi-elastic neutrino events in the GRAIN volume. Sometimes, the photons can be produced in the region between the mask and the sensors, \say{dazzling} the camera. Since the reconstruction technique can only exploit \say{non-dazzled} cameras, for a more accurate estimate of the photon source distribution, it is necessary to exclude the dazzled ones. To this end, we implemented a deep-learning algorithm that classifies the cameras. 
After obtaining the voxel score, applying a selection cut and voxel distribution clustering, we employed the local principal curve algorithm to identify the set of points (referred to as \lpc points) that best approximate this distribution. Subsequently, employing the Hough transform method, we identified the collinear points from these \lpc points and performed linear fitting on them.
As a final step, we estimated the direction of the produced particles. The performance of the reconstruction process was assessed comparing the results with the Monte Carlo truth.

In chapter 1 an overview of the main properties of neutrinos, their interactions  and of the oscillation phenomenon is presented; furthermore, the main neutrino experiments and the open questions are explained. Chapter 2 presents the DUNE experiment with the detector design, its physics motivations, and sensitivity. In chapter 3 a detailed description of the SAND detector and its physics goals, with a particular focus on the GRAIN sub-detector, is provided. In chapter 4, the description of the pre-existing simulation and photon distribution reconstruction is presented. In chapter 5 the implementation and the results of a Deep Learning algorithm to filter GRAIN data are discussed. In chapter 6 the track reconstruction process and fitting method are presented. Finally, the reconstructed particle directions are compared to the corresponding Monte Carlo truth.

