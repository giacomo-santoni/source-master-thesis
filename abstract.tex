The Deep Underground Neutrino Experiment (DUNE) is a next-generation long-baseline accelerator experiment under construction in the United States which aims to address the open questions in neutrino physics, by measuring several undetermined parameters, such as the mass ordering and the CP violating phase. 

DUNE will consist of a Near and a Far Detector complex, $\sim 1300$ km apart. One of three sub-components of the Near Detector complex is the SAND apparatus, which will include GRanular Argon for Interaction of Neutrinos (GRAIN). GRAIN is a novel liquid Argon detector that aims at imaging neutrino interactions with scintillation light detected through an optical readout system based on \emph{coded aperture cameras}, which allow to obtain a voxelized distribution of the photon emission.

This work aims to assess the performance of a track finding algorithm for the reconstruction of charged-current quasi-elastic neutrino interactions in the GRAIN volume. A convolutional neural network algorithm is implemented to filter the cameras suitable for the voxel reconstruction, improving the dataset purity. From the 3D reconstructed voxel distribution a sequence of algorithms has been optimized to obtain track candidates. A comparison between the reconstructed tracks and the Monte Carlo truth is carried out obtaining a good match of the vertex position with an excellent estimate of the track direction. 